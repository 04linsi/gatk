\documentclass[nofootinbib,amssymb,amsmath]{revtex4}
\usepackage{mathtools}
\usepackage{amsthm}
\usepackage{algorithm}
\usepackage{algpseudocode}
\usepackage{lmodern}
\usepackage{graphicx}
\usepackage{color}
\usepackage{bm}

%Put an averaged random variable between brackets
\newcommand{\ave}[1]{\left\langle #1 \right\rangle}

\newcommand{\vzero}{{\bf 0}}
\newcommand{\vI}{{\bf I}}
\newcommand{\vb}{{\bf b}}
\newcommand{\vd}{{\bf d}}
\newcommand{\vf}{{\bf f}}
\newcommand{\vc}{{\bf c}}
\newcommand{\vv}{{\bf v}}
\newcommand{\vz}{{\bf z}}
\newcommand{\vn}{{\bf n}}
\newcommand{\vm}{{\bf m}}
\newcommand{\vG}{{\bf G}}
\newcommand{\vQ}{{\bf Q}}
\newcommand{\vM}{{\bf M}}
\newcommand{\vW}{{\bf W}}
\newcommand{\vX}{{\bf X}}
\newcommand{\vPsi}{{\bf \Psi}}
\newcommand{\vSigma}{{\bf \Sigma}}
\newcommand{\vlambda}{{\bf \lambda}}
\newcommand{\vpi}{{\bf \pi}}
\newcommand{\valpha}{{\bm{\alpha}}}
\newcommand{\vbeta}{{\bm{\beta}}}
\newcommand{\vomega}{{\bm{\omega}}}
\newcommand{\vLambda}{{\bf \Lambda}}
\newcommand{\vA}{{\bf A}}

\newcommand{\code}[1]{\texttt{#1}}

\newtheorem{lemma}{Lemma}
\newtheorem{corollary}{Corollary}

\def\SL#1{{\color [rgb]{0,0,0.8} [SL: #1]}}
\def\DB#1{{\color [rgb]{0,0.8,0} [DB: #1]}}

\newcommand{\HOM}{$\mathsf{Hom}$}
\newcommand{\HET}{$\mathsf{Het}$}
\newcommand{\REF}{$\mathsf{Ref}$}
\newcommand{\epss}{\varepsilon}

\begin{document}

\title{Mathematical Notes on Mutect}
\author{David Benjamin}
\email{davidben@broadinstitute.org}
\affiliation{Broad Institute, 75 Ames Street, Cambridge, MA 02142}

\date{\today}

\maketitle

\section{Introduction}\label{introduction}

We have a set of potential somatic alleles and read-allele likelihoods $\ell_{ra} \equiv P({\rm read~}r|{\rm allele~}a)$.  We don't know which alleles are real somatic alleles and so we must compute, for each subset $\mathbb{A}$ of alleles, the likelihood that the reads come from $\mathbb{A}$.  A simple model for this likelihood is as follows: each read $r$ is associated with a latent indicator vector $\vz_r$ with one-hot encoding $z_{ra} = 1$ iff read $r$ came from allele $a \in \mathbb{A}$.  The conditional probability of the reads $\mathbb{R}$ given their allele assignments is
\begin{equation}
P( \mathbb{R} | \vz, \mathbb{A}) = \prod_{r \in \mathbb{R}} \prod_a \ell_{ra}^{z_{ra}}.
\end{equation}
The alleles are not equally likely because there is a latent vector $\vf$ of allele fractions -- $f_a$ is the allele fraction of allele $a$.  Since the components of $\vf$ sum to one it is a categorical distribution and can be given a Dirichlet prior,
\begin{equation}
P(\vf) = {\rm Dir}(\vf | \valpha).
\end{equation}
Then $f_a$ is the prior probability that a read comes from allele $a$ and thus the conditional probability of the indicators $\vz$ given the allele fractions $\vf$ is
\begin{equation}
P(\vz | \vf) = \prod_r \prod_a f_a^{z_{ra}}.
\end{equation}
The full-model likelihood is therefore
\begin{equation}
\mathbb{L}(\mathbb{A}) = P(\mathbb{R}, \vz, \vf | \mathbb{A}) = {\rm Dir}(\vf | \valpha) \prod_a  \prod_r \left( f_a \ell_{ra}\right)^{z_{ra}}.
\label{full_likelihood}
\end{equation}
And the marginalized likelihood of $\mathbb{A}$, that is, the model evidence for allele subset $\mathbb{A}$, is
\begin{equation}
P(\mathbb{R} | \mathbb{A}) = \sum_\vz \int d \vf \, {\rm Dir}(\vf | \valpha) \prod_a  \prod_r \left( f_a \ell_{ra}\right)^{z_{ra}},
\label{evidence}
\end{equation}
where the integral is over the probability simplex $\sum_a f_a = 1$.

The integral over $\vf$ is the normalization constant of a Dirichlet distribution and as such we can simply look up its formula.  However, the sum over all values of $\vz$ for all reads has exponentially many terms.  We will get around this difficulty by handling $\vz$ with a mean-field approximation in which we factorize the likelihood as $\mathbb{L} \approx q(\vz) q(\vf)$.  This approximation is exact in two limits: first, if there are many reads, each allele is associated with many reads and therefore the Law of Large Numbers causes $\vf$ and $\vz$ to become uncorrelated.  Second, if the allele assignments of reads are obvious $\vz_r$ is effectively not a random variable at all (there is no uncertainty as to which of component is non-zero) and also becomes uncorrelated with $\vf$.

In the variational Bayesian mean-field formalism the value of $\vf$ that $\vz$ ``sees'' is the expectation of $\log \mathbb{L}$ with respect to $q(\vf)$ and vice versa.  That is,
\begin{equation}
q(\vf) \propto {\rm Dir}(\vf | \valpha) \prod_a  \prod_r f_a^{\bar{z}_{ra}} \propto {\rm Dir}(\vf | \valpha + \sum_r \bar{\vz}_r),
\label{qf}
\end{equation}
where $\bar{z}_{ra} \equiv E_q \left[ z_{ra} \right]$, and
\begin{equation}
q(\vz_r) = \prod_a \left( \tilde{f}_a \ell_{ra}\right)^{z_{ra}}, \tilde{f}_a = \exp E[\ln f_a]
\end{equation}
Because $q(\vz)$ is categorical and $q(\vf)$ is Dirichlet\footnote{Note that we didn't \textit{impose} this in any way.  It simply falls out of the mean field equations.} the necessary mean fields are easily obtained and we have
\begin{equation}
\bar{z}_{ra} = \frac{\tilde{f}_a \ell_{ra}}{\sum_{a^\prime} \tilde{f}_{a^\prime} \ell_{ra^\prime}}
\label{z_mean_field}
\end{equation}
and
\begin{equation}
\ln \tilde{f}_a = \psi(\alpha_a + \sum_r \bar{z}_{ra}) - \psi(\sum_{a^\prime} \alpha_{a^\prime} + N)
\label{f_mean_field}
\end{equation}
where $\psi$ is the digamma function and $N$ is the number of reads.  To obtain $q(\vz)$ and $q(\vf)$ we iterate Equations \ref{z_mean_field} and \ref{f_mean_field} until convergence.  A very reasonable initialization is to set $\bar{z}_{ra} = 1$ if $a$ is the most likely allele for read $r$, 0 otherwise.  Having obtained the mean field of $\vz$, we would like to plug it into Eq \ref{evidence}.  We can't do this directly, of course, because Eq \ref{evidence} says nothing about our mean field factorization.  Rather, we need the variational approximation (Bishop's Eq 10.3) to the model evidence, which is
\begin{align}
\ln P(\mathbb{R} | \mathbb{A}) \approx& \sum_{\vz} \int d \vf q(\vz) q(\vf) \left[ \ln P(\mathbb{R}, \vz, \vf | \mathbb{A}) - \ln q(\vz) - \ln q(\vf) \right] \\
=& E_q \left[ \ln P(\mathbb{R}, \vz, \vf | \mathbb{A}) \right] - E_q \left[ \ln q(\vz) \right] - E_q \left[ \ln q(\vf) \right]. \label{lagrangian}
\end{align}
Before we proceed, let's introduce some notation.  First, from Eq \ref{qf} the posterior $q(\vf)$ is
\begin{equation}
q(\vf) = {\rm Dir}(\vf | \vbeta), \quad \vbeta = \valpha + \sum_r \bar{\vz}_r.
\end{equation}
Second, let's define the log normalization constant of a Dirichlet distribution as $g$ so that
\begin{equation}
\ln {\rm Dir}(\vf | \vomega) = g(\vomega) + \sum_a (\omega_a - 1) \ln f_a, \quad g(\vomega) = \ln \Gamma(\sum_a \omega_a) - \sum_a \ln \Gamma(\omega_a).
\end{equation}
Finally, define the Dirichlet mean log (aka ``that digamma stuff") as $h$:
\begin{equation}
E_{\rm Dir(\vf | \vomega)} \left[ \ln f_a \right] = \psi(\omega_a) - \psi(\sum_{a^\prime} \omega_{a^\prime}) \equiv h_a(\vomega).
\end{equation}

The log of Eq \ref{full_likelihood} is
\begin{equation}
\ln P(\mathbb{R}, \vz, \vf | \mathbb{A}) = g(\valpha) + \sum_a (\alpha_a - 1) \ln f_a + \sum_{ra} z_{ra} (\ln f_a + \ln \ell_{ra}).
\end{equation}
and thus the first term in Eq \ref{lagrangian} is
\begin{align}
E_q \left[ \ln P(\mathbb{R}, \vz, \vf | \mathbb{A}) \right] =& g(\valpha) + \sum_a (\alpha_a - 1) h_a(\vbeta) + \sum_{ra}\bar{z}_{ra} \left( h_a(\vbeta) + \ln \ell_{ra} \right) \\
=& g(\valpha) + \sum_a (\beta_a - 1) h_a(\vbeta) + \sum_{ra}\bar{z}_{ra} \ln \ell_{ra}, \label{first_term}
\end{align}
where we used the relationship $\vbeta = \valpha + \sum_r \bar{\vz}_r$.

The second term in Eq \ref{lagrangian} is
\begin{align}
- E_q \left[ \ln q(\vz) \right] = - \sum_{ra} \bar{z}_{ra} \ln \bar{z}_{ra} \label{second_term}.
\end{align}

The third term in Eq \ref{lagrangian} is
\begin{align}
- E_q \left[ \ln q(\vf) \right] = -g(\vbeta) - \sum_a (\beta_a - 1) E_q [\ln f_a] = -g(\vbeta) - \sum_a (\beta_a - 1) h_a(\vbeta) \label{third_term}.
\end{align}

Adding Eqs \ref{first_term}, \ref{second_term}, and \ref{third_term} and noting the cancellation between parts of Eqs \ref{first_term} and \ref{third_term} we obtain
\begin{equation}
\ln P(\mathbb{R} | \mathbb{A}) \approx g(\valpha) - g(\vbeta) +  \sum_{ra} \bar{z}_{ra} \left( \ln \ell_{ra} - \ln \bar{z}_{ra} \right).
\end{equation}

We now have the model evidence for allele subset $\mathbb{A}$.  This lets us choose which alleles are true somatic variants.  It also lets us make calls on somatic loss of heterozygosity events.  Furthermore, instead of reporting max-likelihood allele fractions as before, we may emit the parameters of the Dirichlet posterior $q(\vf)$, which encode both the maximum likelihood allele fractions and their uncertainty.

\end{document}